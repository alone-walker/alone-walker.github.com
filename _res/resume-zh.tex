\documentclass[10pt,a4paper, noblankpage]{moderncv}

% moderncv themes
\moderncvtheme[blue]{classic}                % idem

% character encoding
\usepackage{fontspec, xunicode, xltxtra}
\setmainfont[Mapping=tex-text]{Microsoft YaHei}
\setsansfont[Mapping=tex-text]{Microsoft YaHei}
\XeTeXlinebreaklocale "zh"
\widowpenalty=10000

% adjust the page margins
\usepackage[scale=0.85]{geometry}
\AtBeginDocument{\recomputelengths}

% personal data
\firstname{刘卫}\familyname{}
\title{\small \ \ KISS \& DRY}
%\title{简历}
\address{上海市浦东新区}{金葵新城}
\mobile{13701680238}
%\phone{010-12345678}
\email{wartalker@gmail.com}
%\extrainfo{群众}
%\photo[64pt]{wartalker.jpg}

\nopagenumbers{}

%----------------------------------------------------------------------------------
%            content
%----------------------------------------------------------------------------------
\begin{document}
\maketitle
\section{技术}
\cvline{}{嵌入式系统c开发}
\cvline{}{linux内核: 文件系统, tcp/ip协议栈, 网卡驱动}
\cvline{}{扎实的算法与数据结构基础}
\cvline{}{opengl图形处理}
%%\cvline{}{}

\section{语言与工具}
\cvline{}{c, c++, python,java, lua}
\cvline{}{gcc, gdb,make, vim, visual studio c++}
%%\cvline{}{}

\section{教育背景}
\cvline{}{西安电子科技大学 机械设计及制造专业} 
%%\cvline{}{}

\section{工作经历}
\cvline{}{}
\cventry{2011--2013}{敏锐德}{高级工程师}{}{}{}
\cvline{}{MTK手机平台下java虚拟机的移植与维护,实现了文件、网络、nfc、gps等模块}
\cvline{}{技术: embedded system c programming, JVM, j2me}
\cvline{}{}
\cventry{2009--2010}{中科方德}{高级工程师}{}{}{}
\cvline{}{扩展开发了postfix邮件服务的邮件过滤、加密签名等功能}
\cvline{}{PKI USB Token的RSA加解密算法实现}
\cvline{}{技术: linux c programming, glibc, rsa, openssl}
\cvline{}{}
\cventry{2005--2009}{中国数码}{工程师}{}{}{}
\cvline{}{金融网络通讯协议实现}
\cvline{}{html数据解析}
\cvline{}{http代理服务器}
\cvline{}{技术: c++, stl, boost, socket, libxml, mmap, http }
\cvline{}{}
\cventry{2002--2005}{力神电池股份有限公司}{工程师}{}{}{}
\cvline{}{自动生产线的设备维护,熟悉欧姆龙、三菱等多种PLC}
\cvline{}{}

\section{读过的开源项目}
\cvline{}{stl, xv6, lua, linux kernel, goagent}
%%\cvline{}{}
\section{个人项目}
\cvline{}{Github: \href{https://github.com/wartalker} {https://github.com/wartalker}}

\end{document}

%% end of file `resume.tex'.
